\phantomsection
\chapter*{毕业设计小结}
\addcontentsline{toc}{chapter}{\fHei 毕业设计小结}

本次毕业设计是大学阶段的最后一次挑战。期间想了很多优化的方法和思路,但是由于自身对问题认识的不够深刻,提出了一些不切实际的优化思路,并且在这些不靠谱的想法上浪费了很多时间,其中包括企图通过超图分割算法进行任务图的分割,然后绑定到NUMA结点上,来提升内存的带宽,减少总线的冲突,后来经过实践,被证实是低效的。后来我有发现了一个名为taskflow的计算框架,可以在运行时构建任务依赖图,并且使用work stealing调度算法进行并行地执行这张图,性能好于tpp和clik,但是基于该框架实现了SpTRSV算法之后,效果不是很理想。

虽然这次毕业设计对我来说是一次艰巨的挑战,但是也意义非凡。通过这次毕业设计我学习与回顾了本科所学的计算机知识,主要包括计算机组成原理和体系结构、计算机操作系统、编译原理,并将这些知识应用于实践当中,期间有想法不成功、遇到bug时候的挫败感,也有通过自己不断研究、不断实践,逐渐提升算法性能时的成就感。同时这次毕业设计也让我学到了一些科研的基础技能,为应对未来的挑战打下了基础。

总的来说,这次毕业设计是我从一个学习模仿者转变为一个创新研究者所必须经历的机遇与挑战。


\clearpage
\endinput