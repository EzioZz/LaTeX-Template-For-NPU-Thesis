\chapter{总结与展望}


\section{工作总结}

SpTRSV稀疏下三角矩阵求解是现代数值计算中的一个重要算法。是现代科学计算中一个广泛使用的计算核心,在数值模拟计算中,通常会使用迭代法或直接法求解大规模稀疏线性方程组,而SpTRSV的效率直接影响了线性方程组的求解效率,提高SpTRSV算法的性能至关重要。SpTRSV频繁且离散地数据访存、任务之间存在着很强的依赖、需要细粒度的同步、任务之间负载不均衡等特点是并行优化更加难以进行。

TODO:



\section{未来工作展望}

对于SpTRSV算法还有很多可以尝试的优化思路,例如:在任务调度方面,基于手动创建的线程池,并设计一种调度器,通过这个调度器,在线程任务分发即阶段实现任务负载均衡的操作,减少预处理的时间。并且基于这个调度器,通过一定的调度策略减少CPU自旋等待的时间,这个思路来自于GPU的硬件调度器;在访存的优化方面,目前较多的操作需要离散且频繁的从内存进行读取,导致了内存瓶颈。

\endinput