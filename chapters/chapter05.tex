\chapter{总结与展望}


\section{工作总结}

SpTRSV稀疏下三角矩阵求解是现代数值计算中的一个重要算法。是现代科学计算中一个广泛使用的计算核心,在数值模拟计算中,通常会使用迭代法或直接法求解大规模稀疏线性方程组,而SpTRSV的效率直接影响了线性方程组的求解效率,提高SpTRSV算法的性能至关重要。SpTRSV频繁且离散的数据访存、任务之间存在着很强的依赖、需要细粒度的同步、任务之间负载不均衡等特点是并行优化更加难以进行,正是这些特点使得通用且高效的SpTRSV算法仍然是一个颇具挑战性的课题。

本文提出了一个通过减少预处理的消耗,使用原子操作以及自旋等待的方式在运行时维护任务之间的运行顺序。并且结合了鲲鹏920处理器体系结构的特点进行了优化,例如使用了SIMD指令,ARMv8的原子指令,以及一些缓存优化的技巧。最终本文所设计的SpTRSV算法对于任务间依赖相对较少的矩阵能够实现相比于串行算法2倍以上的的加速比,对于任务依赖较为复杂的矩阵,算法的加速性能不够理想。


\section{未来工作展望}

对于SpTRSV算法还有很多可以尝试的优化思路,例如:在任务调度方面,基于手动创建的线程池,并设计一种调度器,通过该调度器在线程任务分发阶段实现任务负载均衡,减少预处理的时间。并且基于该调度器,通过一定的调度策略减少CPU自旋等待的时间,这个思路来自于GPU的硬件调度器;在访存的优化方面,目前较多的操作需要离散且频繁的从内存进行读取,因此,优化内存访问也是提升SpTRSV算法性能的有效方法。

\endinput