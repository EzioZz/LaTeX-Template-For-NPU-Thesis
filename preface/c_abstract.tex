\renewcommand{\baselinestretch}{1.5}
\fontsize{12pt}{13pt}\selectfont

\chapter[摘要]{摘~~~~要}
\markboth{中~文~摘~要}{中~文~摘~要}

稀疏下三角矩阵求解器,是数值计算中的一个重要的方法,在科学计算中有着广泛的应用。由于其离散的数据访存、任务之间存在着很强的依赖、需要细粒度的同步、任务之间负载不均衡等特点,在现代多核系统中,如何提升SpTRSV算法的并行度与扩展性,仍然是一个挑战性的课题。

华为鲲鹏920处理器基于ARMv8架构,最多拥有64个核心,支持ARM NEON 128bits浮点运算。华为鲲鹏系统支持NUMA内存架构,实现最多4个920互联,最高支持256个计算核心。

本文基于华为鲲鹏920芯片,进行稀疏下三角矩阵求解算法的设计与优化。

主要工作包括:
\vspace{-10pt}
\begin{enumerate}
    \item 设计并实现了一套高效的稀疏下三角矩阵求解算法。
    \item 针对华为鲲鹏920处理器众核体系结构以及ARMv8指令集的特点进行了优化。
    \item 采用ARM NEON指令,对向量乘法部分进行了SIMD的优化。
    \item 结合处理器NUMA架构的特性,通过分配矩阵数据的存储,充分利用内存带宽。
\end{enumerate}
\vspace{-10pt}

\vspace{1em}
\noindent {\fHei 关键词:} \quad 稀疏下三角矩阵求解器, 鲲鹏920, 众核优化, 并行算法

\clearpage
\endinput
