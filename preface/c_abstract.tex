\renewcommand{\baselinestretch}{1.5}
\fontsize{12pt}{13pt}\selectfont

\chapter[摘要]{摘~~~~要}
\markboth{中~文~摘~要}{中~文~摘~要}

稀疏下三角矩阵求解器(sparse triangular solver,SpTRSV),是数值计算中的一个重要的方法,在科学计算中有着广泛的应用。由于其离散的数据访存、任务之间存在着很强的依赖、需要细粒度的同步、任务之间负载不均衡等特点,在现代多核系统中,如何提升SpTRSV算法的并行度与扩展性,仍然是一个挑战性的课题。

华为鲲鹏920处理器基于ARMv8架构,最多拥有64个核心,支持ARM NEON 128bits浮点运算。华为鲲鹏系统支持NUMA内存架构,实现最多4个920互联,最高支持256个计算核心。

本文基于华为鲲鹏920芯片,进行稀疏下三角矩阵求解算法的设计与优化。

主要工作包括:% TODO:以下四点都要详细说明
\vspace{-10pt}
\begin{enumerate}
    \item 设计并实现了一套高效的稀疏下三角矩阵求解算法。与传统基于level-sets方法不同的是,该算法并没有采用构造任务依赖图(TDG)大幅减少了矩阵预处理的时间,使用原子操作以及自旋等待在运行时维护任务顺序。根据测试该算法相比于串行方法能够获得约2倍以上加速比, 并且在一些稀疏矩阵中能够取得比基于level-sets算法更好的性能提升。
    \item 针对华为鲲鹏920处理器众核体系结构以及ARMv8架构的特点进行了优化。在原有算法的基础上通过结合体系结构的优化(例如:消除为伪共享、使用ARMv8原子指令、CPU松弛等技术)能够进一步获得约10\%的性能提升。
    \item 采用ARM NEON指令,对向量乘法部分进行了SIMD的优化。对于float数据类型,进行了SIMD指令的优化,执行一条NEON乘法指令,相当于进行了四次乘法运算。
    \item 结合处理器NUMA架构的特性,对于规模较小的矩阵将任务绑定在单一的节点上,通过减少结点间的通信延迟,进行性能的优化。对于规模较大的矩阵,分配矩阵数据到不同节点上,通过充分利用内存带宽和核心数量,能够获得相比于单NUMA结点约15\%的性能提升。
\end{enumerate}
\vspace{-10pt}

\vspace{1em}
\noindent {\fHei 关键词:} \quad 稀疏下三角矩阵求解器, 鲲鹏920, 众核优化, 并行算法

\clearpage
\endinput
